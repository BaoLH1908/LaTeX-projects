\documentclass[10pt, a4paper]{article}
\usepackage[paperheight=30cm,paperwidth=20cm,includehead,nomarginpar,textwidth=17cm,textheight=25cm,headheight=4mm]{geometry}
\usepackage{fancyhdr}
\usepackage[utf8]{vietnam}
\usepackage[english]{babel}
\usepackage{xcolor,amsmath,amssymb,amsfonts}
\usepackage[most]{tcolorbox}
\usepackage{wrapfig}
\begin{document}
	% Set the page style to "fancy"...
	\pagestyle{fancy}
	%... then configure it.
	\fancyhead{} % clear all header fields
	\fancyhead[R]{\textbf{Trường Đại học Khoa học Tự nhiên, ĐHQG-HCM\\Bộ môn Xác suất - Thống kê, Khoa Toán - Tin học}}
	\fancyhead[L]{\color{red}\textbf{\LaTeX~by Lê Hoàng Bảo}}
	\fancyfoot{} % clear all footer fields
	\fancyfoot[C]{\textbf\thepage}
	\fancyfoot[L]{\small LÝ THUYẾT THỐNG KÊ}
	\fancyfoot[R]{Mã môn học: MTH10404}
	\renewcommand{\headrulewidth}{0.6pt}
	\renewcommand{\footrulewidth}{0.6pt}
\begin{center}
	\textbf{\color{purple}LỜI GIẢI ĐỀ THI CUỐI HỌC KÌ I LÝ THUYẾT THỐNG KÊ (2023 - 2024)}
\end{center}
\vspace{1mm}
\begin{tcolorbox}[enhanced,colback=blue!5!white,colframe=blue!75!black,sharp corners=all,shadow={0mm}{0mm}{-1.5mm}%
{fill=blue!75!red,opacity=0.3},title=\textbf{Bài 1}]
Đo chỉ số chất béo $X$ (đơn vị: \%) trong sữa bò của 125 con bò thuộc một giống bò sữa lai mới của Hà Lan, ta được bảng số liệu sau:\begin{center}
\begin{tabular}{c|c|c|c|c|c|c|c}
	$X$&3,5&3,8&4,5&5,2&5,6&6,4&6,8\\ \hline
	$n_i$&2&8&35&40&20&15&5
\end{tabular}
\end{center}
Giả thiết rằng $X$ có phân phối chuẩn.\\\\
\color{red}a) \color{black}Tìm khoảng tin cậy 99\% cho trung bình chỉ số chất béo trong sữa giống bò lai trên.
\end{tcolorbox}
\begin{center}
	\color{blue}\underline{Lời giải:}
\end{center}
Đối tượng khảo sát có cỡ mẫu lớn $(125>30)$ và chưa biết phương sai tổng thể.\\\\
Ta có trung bình mẫu: $$\overline x=\frac1n\sum_{i=1}^7n_ix_i=\frac{2.3,5+8.3,8+35.4,5+40.5,2+20.5,6+15.6,4+5.6,8}{125}=5,1592$$
và phương sai mẫu: $$s^2=\frac{1}{n-1}\sum_{i=1}^7n_i(x_i-\overline x)^2\approx0,6129$$
Độ tin cậy $1-\alpha=99\%\Rightarrow\alpha=0,01$ nên ta có phân vị $z_{1-\frac\alpha2}=z_{0,995}=2,58$.\\\\
Dung sai: $$\varepsilon=z_{1-\frac\alpha2}\frac{s}{\sqrt n}=2,58\sqrt{\frac{0,6129}{125}}\approx0,1806$$
Vậy khoảng tin cậy 99\% cho trung bình chỉ số chất béo trong sữa giống bò lai trên là: $$\mu\in[\overline x-\varepsilon,\overline x+\varepsilon]=[5,1592-0,1806;5,1592+0,1806]=[4,9786;5,3398]$$
\vspace{0.1mm}
\begin{tcolorbox}[enhanced,colback=blue!5!white,colframe=blue!75!black,sharp corners=all,shadow={0mm}{0mm}{-1.5mm}%
{fill=blue!75!red,opacity=0.3}]
\color{red}b) \color{black}Biết trung bình chỉ số chất béo trong sữa giống bò thuần chủng (giống bò cũ) là 4,65. Việc lai tạo có cho trung bình chỉ số chất béo của sữa bò \textbf{tăng lên} hay không, với $\alpha=1\%$?
\end{tcolorbox}
\begin{center}
	\color{blue}\underline{Lời giải:}
\end{center}
Gọi $\mu$ là trung bình chỉ số chất béo trong sữa của giống bò mới.\\\\
"Trung bình chỉ số chất béo trong sữa giống bò thuần chủng (giống bò cũ) là 4,65" $\Rightarrow\mu_0=4,65$.\\\\
"Việc lai tạo có cho trung bình chỉ số chất béo của sữa bò \textbf{tăng lên} hay không" $\Rightarrow H_1:\mu>\mu_0=4,65$.\\\\
Ta tiến hành kiểm định giả thuyết $\begin{cases}
	H_0:\mu=4,65\\
	H_1:\mu>4,65
\end{cases}$\\\\
Thống kê kiểm định: $$z_0=\frac{\overline x-\mu_0}{s/\sqrt n}=\frac{5,1592-4,65}{\sqrt{\frac{0,6129}{125}}}=7,272$$
Ta bác bỏ $H_0$ nếu $z_0>z_{1-\alpha}=z_{0,99}=2,33$. Ta có $z_0=7,272>2,33\Rightarrow$ Bác bỏ $H_0$.\\\\
Vậy với mức ý nghĩa 1\%, việc lai tạo có cho trung bình chỉ số chất béo của sữa bò tăng lên.
\vspace{3mm}
\begin{tcolorbox}[enhanced,colback=blue!5!white,colframe=blue!75!black,sharp corners=all,shadow={0mm}{0mm}{-1.5mm}%
{fill=blue!75!red,opacity=0.3}]
\color{red}c) \color{black}Sữa bò được đánh giá là loại 1 nếu chỉ số chất béo nằm trong khoảng từ 4,0 đến 6,0. Có ý kiến cho rằng \textbf{ít nhất} 70\% lượng sữa bò của giống bò lai mới này thuộc loại 1. Hãy kiểm định ý kiến trên với mức ý nghĩa 5\%. Tính $p$-giá trị.
\end{tcolorbox}
\begin{center}
	\color{blue}\underline{Lời giải:}
\end{center}
Đặt $Y$ là số con bò mà sữa của chúng thuộc loại 1. Ta tính: $$\widehat P=\frac{Y}{n}=\frac{35+40+20}{125}=0,76$$
Đặt $p$ là tỷ lệ lượng sữa của giống bò mới mà thuộc loại 1.\\\\
"Có ý kiến cho rằng \textbf{ít nhất} 70\% lượng sữa bò của giống bò lai mới này thuộc loại 1" $\Rightarrow\begin{cases}
	p_0=0,7\\
	H_0:p\ge p_0=0,7
\end{cases}$\\\\
Ta kiểm tra: $n.p_0=125.0,7=87,5\ge5$ và $n(1-p_0)=37,5\ge5$.\\\\
Ta tiến hành kiểm định giả thuyết $\begin{cases}
	H_0:p\ge0,7\\
	H_1:p<0,7
\end{cases}$\\\\
Thống kê kiểm định: $$z_0=\frac{\widehat P-p_0}{\sqrt{\frac{p_0(1-p_0)}{n}}}=\frac{0,76-0,7}{\sqrt{\frac{0,7.0,3}{125}}}\approx1,464$$
Ta bác bỏ $H_0$ nếu $z_0<-z_{1-\alpha}=-z_{0,95}=-1,64$. Ta có $z_0=1,464>-1,64\Rightarrow$ Không bác bỏ $H_0$.\\\\
Vậy với mức ý nghĩa 5\%, ý kiến trên là đúng.\\\\
Ta tính $p$-giá trị: $p=\Phi(z_0)=\Phi(1,464)=0,9284$.
\vspace{3mm}
\begin{tcolorbox}[enhanced,colback=blue!5!white,colframe=blue!75!black,sharp corners=all,shadow={0mm}{0mm}{-1.5mm}%
{fill=blue!75!red,opacity=0.3},title=\textbf{Bài 2}]
Khảo sát đường kính những thanh thép được sản xuất bởi hai máy dập tự động. Hai mẫu ngẫu nhiên $n=14$ và $m=16$ được chọn, tính được trung bình mẫu và phương sai mẫu lần lượt là $$\overline x=8,72;~s_1^2=0,35;~\overline y=8,68;~s_2^2=0,4.$$ Giả sử dữ liệu chọn từ các tổng thể có phân phối chuẩn và $\sigma_1^2=\sigma_2^2$. Với mức ý nghĩa 5\%, có đủ bằng chứng để khẳng định rằng đường kính trung bình của các thanh thép do hai máy dập tự động này sản xuất là khác nhau hay không?
\end{tcolorbox}
\begin{center}
	\color{blue}\underline{Lời giải:}
\end{center}
Gọi $\mu_1$ và $\mu_2$ lần lượt là đường kính trung bình của các thanh thép do máy dập tự động I và II sản xuất.\\\\
"Đường kính trung bình của các thanh thép do hai máy dập tự động này sản xuất là khác nhau"\\$\Rightarrow H_1:\mu_1-\mu_2\ne D_0=0$.\\\\
Ta tiến hành kiểm định giả thuyết $\begin{cases}
	H_0:\mu_1-\mu_2=0\\
	H_1:\mu_1-\mu_2\ne0
\end{cases}$ với giả sử rằng $\sigma_1^2=\sigma_2^2$.\\\\
Ta tính phương sai mẫu chung: $$s_p^2=\frac{(n-1)s_1^2+(m-1)s_2^2}{n+m-2}=\frac{(14-1).0,35+(16-1).0,4}{14+16-2}\approx0,377$$
Thống kê kiểm định: $$t_0=\frac{\overline x-\overline y-(\mu_1-\mu_2)}{\sqrt{\dfrac{s_p^2}{n}+\dfrac{s_p^2}{m}}}=\frac{8,72-8,68-0}{\sqrt{\dfrac{0,377}{14}+\dfrac{0,377}{16}}}\approx0,178$$
Ta bác bỏ $H_0$ nếu $|t_0|>t_{1-\frac\alpha2}^{n+m-2}=t^{28}_{0,975}=2,0484$. Ta có $|t_0|=0,178<2,0484\Rightarrow$ Không bác bỏ $H_0$.\\\\
Vậy với mức ý nghĩa 5\%, ta không có đủ bằng chứng để khẳng định rằng đường kính trung bình của các thanh thép do hai máy dập tự động này sản xuất là khác nhau.
\vspace{3mm}
\begin{tcolorbox}[enhanced,colback=blue!5!white,colframe=blue!75!black,sharp corners=all,shadow={0mm}{0mm}{-1.5mm}%
{fill=blue!75!red,opacity=0.3},title=\textbf{Bài 3}]
Một nghiên cứu về mối liên hệ giữa tỷ lệ cây xanh $X$ (đơn vị: m$^2$/người) và nhiệt độ trung bình trong mùa hè $Y$ (đơn vị: $^\text{o}$C) được khảo sát tại 12 thành phố (các thành phố này có cùng kiểu khí hậu địa lý), thu được dữ liệu sau: $$\displaystyle\sum_{j=1}^{12}x_j=241;~~~\displaystyle\sum_{j=1}^{12}x_j^2=7281;~~~\displaystyle\sum_{j=1}^{12}x_jy_j=6404;~~~\displaystyle\sum_{j=1}^{12}y_j=341,5;~~~\displaystyle\sum_{j=1}^{12}y_j^2=9813,25$$
\color{red}a) \color{black}Tìm phương trình đường thẳng hồi quy tuyến tính bằng phương pháp bình phương nhỏ nhất.
\end{tcolorbox}
\begin{center}
	\color{blue}\underline{Lời giải:}
\end{center}
Bằng phương pháp bình phương nhỏ nhất, ta tính hệ số góc: $$\hat\beta_1=\frac{S_{xy}}{S_{xx}}=\frac{\displaystyle\sum_{j=1}^{12}x_jy_j-\dfrac1n\displaystyle\sum_{j=1}^{12}x_j\displaystyle\sum_{j=1}^{12}y_j}{\displaystyle\sum_{j=1}^{12}x_j^2-\dfrac1n\left(\displaystyle\sum_{j=1}^{12}x_j\right)^2}=\frac{6404-\dfrac{1}{12}\cdot241\cdot341,5}{7281-\dfrac{1}{12}\cdot241^2}\approx-0,1862$$
và hệ số chặn: $$\hat\beta_0=\overline y-\hat\beta_1\overline x=\dfrac1n\displaystyle\sum_{j=1}^{12}y_j+0,186\cdot\dfrac1n\displaystyle\sum_{j=1}^{12}x_j=\dfrac{1}{12}\cdot341,5+\frac{0,1862}{12}\cdot241\approx32,1979$$
Vậy phương trình đường thẳng hồi quy tuyến tính là: $$\hat y=\hat\beta_0+\hat\beta_1x=32,1979-0,1862x$$
\vspace{0.1mm}
\begin{tcolorbox}[enhanced,colback=blue!5!white,colframe=blue!75!black,sharp corners=all,shadow={0mm}{0mm}{-1.5mm}%
{fill=blue!75!red,opacity=0.3}]
\color{red}b) \color{black}Nếu một thành phố (cùng kiểu khí hậu này) có tỷ lệ cây xanh là $x_0=24$ m$^2$/người, dựa vào đường thẳng hồi quy tìm được ở câu a, hãy dự đoán nhiệt độ trung bình trong mùa hè của thành phố đó.
\end{tcolorbox}
\begin{center}
	\color{blue}\underline{Lời giải:}
\end{center}
Với tỷ lệ cây xanh là $x_0=24$ m$^2$/người thì nhiệt độ trung bình là: $$\hat y=32,1938-0,1862x_0=32,1979-0,1862.24=27,7291~(^\text{o}\text{C})$$
\vspace{0.1mm}
\begin{tcolorbox}[enhanced,colback=blue!5!white,colframe=blue!75!black,sharp corners=all,shadow={0mm}{0mm}{-1.5mm}%
{fill=blue!75!red,opacity=0.3}]
\color{red}c) \color{black}Tính hệ số xác định $R^2$ và nhận xét về mối liên hệ tuyến tính giữa tỷ lệ cây xanh và nhiệt độ trung bình trong mùa hè.
\end{tcolorbox}
\begin{center}
	\color{blue}\underline{Lời giải:}
\end{center}
Ta có: \begin{gather*}
	SSR=\hat\beta_1S_{xy}=\hat\beta_1\left(\sum_{j=1}^{12}x_jy_j-\frac1n\sum_{j=1}^{12}x_j\sum_{j=1}^{12}y_j\right)=-0,1862\left(6404-\dfrac{1}{12}\cdot241\cdot341,5\right)\approx84,6201\\
	SST=S_{yy}=\sum_{j=1}^{12}y_j^2-\frac1n\left(\sum_{j=1}^{12}y_j\right)^2=9813,25-\frac{1}{12}(341,5)^2\approx94,7291
\end{gather*}
Hệ số tương quan: $R^2=\dfrac{SSR}{SST}=\dfrac{84,6201}{94,7291}=0,8932$.\\\\
Vậy tỷ lệ cây xanh và nhiệt độ trung bình trong mùa hè có mối liên hệ tuyến tính mạnh.
\vspace{3mm}
\begin{tcolorbox}[enhanced,colback=blue!5!white,colframe=blue!75!black,sharp corners=all,shadow={0mm}{0mm}{-1.5mm}%
{fill=blue!75!red,opacity=0.3},title=\textbf{Bài 4}]
Một bác sĩ dinh dưỡng nghiên cứu một chế độ ăn kiêng và tập thể dục mới để làm giảm lượng đường trong máu của các bệnh nhân bị bệnh tiểu đường. 10 bệnh nhân bị bệnh tiểu đường được chọn để thử nghiệm chương trình này. Bảng số liệu sau cho biết lượng đường trong máu trước và sau khi các bệnh nhân tham gia chương trình:\begin{center}
\begin{tabular}{c|cccccccccc}
	Trước&268&225&252&192&308&228&246&298&230&185\\ \hline
	Sau&106&185&224&110&204&100&212&176&194&205
\end{tabular}
\end{center}
Số liệu trên có đủ bằng chứng để kết luận rằng chế độ ăn kiêng và tập thể dục mới này có tác dụng làm giảm lượng đường trong máu hay không? Sử dụng mức ý nghĩa $\alpha=5\%$.
\end{tcolorbox}
\begin{center}
	\color{blue}\underline{Lời giải:}
\end{center}
Đối tượng khảo sát có cỡ mẫu nhỏ $(10<30)$.\\\\
Ta có độ sai khác $d_i$ giữa mỗi cặp trong quan trắc là: \begin{center}
\begin{tabular}{c|cccccccccc}
	Trước&268&225&252&192&308&228&246&298&230&185\\ \hline
	Sau&106&185&224&110&204&100&212&176&194&205\\ \hline
	$d_i$&162&40&28&82&104&128&34&122&36&-20
\end{tabular}
\end{center}
Ta tính trung bình mẫu: $$\overline d=\frac{1}{n}\sum_{i=1}^nd_i=\frac{1}{10}\sum_{i=1}^{10}d_i=71,6$$
và phương sai mẫu: $$s_d^2=\frac{1}{n-1}\sum_{i=1}^n(d_i-\overline d)^2=\frac19\sum_{i=1}^{10}(d_i-71,6)^2=3224,71$$
Ta tiến hành kiểm định giả thuyết: $$\begin{cases}
H_0:\mu_d\le d_0=0~~\text{(chế độ ăn kiêng và tập thể dục mới không có tác dụng)}\\
H_1:\mu_d>d_0=0~~\text{(chế độ ăn kiêng và tập thể dục mới có tác dụng)}
\end{cases}$$
Thống kê kiểm định: $$t_0=\frac{\overline d-d_0}{s_d/\sqrt n}=\frac{71,6-0}{\sqrt{\frac{3224,71}{10}}}\approx3,9871$$
Ta bác bỏ $H_0$ nếu $t_0>t^{n-1}_{1-\alpha}=t^9_{0,95}=1,8331$. Ta có $t_0=3,9871>1,8331\Rightarrow$ Bác bỏ $H_0$.\\\\
Vậy với mức ý nghĩa 5\%, chế độ ăn kiêng và tập thể dục mới này có tác dụng làm giảm lượng đường trong máu.
\begin{tcolorbox}[enhanced,colback=blue!5!white,colframe=blue!75!black,sharp corners=all,shadow={0mm}{0mm}{-1.5mm}%
{fill=blue!75!red,opacity=0.3},title=\textbf{Bài 5}]
Cho một tổng thể tuân theo phân phối xác suất với trung bình $\mu$ và phương sai $\sigma^2$ với $\sigma>0$. Xét mẫu ngẫu nhiên $(X_1,X_2,X_3)$ được chọn từ tổng thể này. Với $0<a<1$, đặt: $$M_a=\dfrac12[X_1+aX_2+(1-a)X_3]$$
\color{red}a) \color{black}Chứng tỏ $M_a$ là một ước lượng không chệch của $\mu$.
\end{tcolorbox}
\begin{center}
	\color{blue}\underline{Lời giải:}
\end{center}
Vì $(X_1,X_2,X_3)$ được chọn từ tổng thể trên nên: $$\mathbb E(X_1)=\mathbb E(X_2)=\mathbb E(X_3)=\mu,~~\text{Var}(X_1)=\text{Var}(X_2)=\text{Var}(X_3)=\sigma^2$$
Ta có: \begin{align*}
	\mathbb E(M_a)&=\mathbb E\left[\frac12\big(X_1+aX_2+(1-a)X_3\big)\right]\\
	&=\frac12\big[\mathbb E(X_1)+a\mathbb E(X_2)+(1-a)\mathbb E(X_3)\big]\\
	&=\frac12\big[\mu+a\mu+(1-a)\mu\big]\\
	&=\frac12\cdot2\mu=\mu
\end{align*}
$\Rightarrow\text{bias}(M_a)=\mathbb E(M_a)-\mu=\mu-\mu=0$\\\\
Vậy $M_a$ là một ước lượng không chệch cho $\mu$.
\vspace{3mm}
\begin{tcolorbox}[enhanced,colback=blue!5!white,colframe=blue!75!black,sharp corners=all,shadow={0mm}{0mm}{-1.5mm}%
{fill=blue!75!red,opacity=0.3}]
\color{red}b) \color{black}Tính phương sai của $M_a$.
\end{tcolorbox}
\begin{center}
	\color{blue}\underline{Lời giải:}
\end{center}
Phương sai của $M_a$: \begin{align*}
	\text{Var}(M_a)&=\text{Var}\left[\frac12\big(X_1+aX_2+(1-a)X_3\big)\right]\\
	&=\frac14\big[\text{Var}(X_1)+a^2\text{Var}(X_2)+(1-a)^2\text{Var}(X_3)\big]\\
	&=\frac14\big[\sigma^2+a^2\sigma^2+(1-2a+a^2)\sigma^2\big]\\
	&=\frac14\big(2\sigma^2-2a\sigma^2+2a^2\sigma^2\big)\\
	&=\frac{1-a+a^2}{2}\sigma^2
\end{align*}
\vspace{2mm}
\begin{tcolorbox}[enhanced,colback=blue!5!white,colframe=blue!75!black,sharp corners=all,shadow={0mm}{0mm}{-1.5mm}%
{fill=blue!75!red,opacity=0.3}]
\color{red}c) \color{black}Trung bình bình phương sai số (MSE) của $M_a$ được cho bởi MSE$(M_a,\mu)=\mathbb E\left[(M_a-\mu)^2\right]$. Tìm giá trị của $a$ sao cho $M_a$ là ước lượng tốt nhất theo nghĩa cực tiểu hóa trung bình bình phương sai số.
\end{tcolorbox}
\begin{center}
	\color{blue}\underline{Lời giải:}
\end{center}
Ta có: \begin{align*}
	\text{MSE}(M_a,\mu)&=\mathbb E\left[(M_a-\mu)^2\right]\\
	&=\mathbb E\left(M_a^2-2M_a\mu+\mu^2\right)\\
	&=\mathbb E(M_a^2)-2\mu\mathbb E(M_a)+\mu^2\\
	&=\text{Var}(M_a)+[\mathbb E(M_a)]^2-2\mu\mathbb E(M_a)+\mu^2\\
	&=\frac{1-a+a^2}{2}\sigma^2+\mu^2-2\mu\mu+\mu^2\\
	&=\frac{1-a+a^2}{2}\sigma^2
\end{align*}
Để cực tiểu hóa MSE, ta cần tìm giá trị $a$ sao cho $f(a)=1-a+a^2$ đạt GTNN.\\\\
Ta tính $f'(a)=2a-1$ và cho $f(a)=0\Rightarrow a=\dfrac12$\\\\
Ta tiếp tục tính $f''\left(\dfrac12\right)=2>0$\\\\
Vậy với $a=\dfrac12$ thì $M_a$ là ước lượng tốt nhất.
\vspace{2mm}
\begin{tcolorbox}[enhanced,colback=blue!5!white,colframe=blue!75!black,sharp corners=all,shadow={0mm}{0mm}{-1.5mm}%
{fill=blue!75!red,opacity=0.3},title=\textbf{Bài 6}]
Cho hai tổng thể độc lập với nhau, trong đó:\\\\
$\bullet~$tổng-thể-1 có trung bình $\mu_1$ và phương sai $\sigma_1^2$ với $\sigma_1>0$. Mẫu ngẫu nhiên với cỡ mẫu $n$\\ được chọn từ tổng-thể-1 có trung bình mẫu là $\overline X$ và phương sai mẫu là $S_1^2$.\\\\
$\bullet~$tổng-thể-2 có trung bình $\mu_2$ và phương sai $\sigma_2^2$ với $\sigma_2>0$. Mẫu ngẫu nhiên với cỡ mẫu $m$\\ được chọn từ tổng-thể-2 có trung bình mẫu là $\overline Y$ và phương sai mẫu là $S_2^2$.\\\\
\color{red}a) \color{black}Biết sai số chuẩn (standard error) của ước lượng $\hat\theta$ được cho bởi $$\text{SE}\big(\widehat\theta\big):=\sqrt{\text{Var}\big(\widehat\theta\big)}$$ Tính sai số chuẩn SE$(\overline X-\overline Y)$.
\end{tcolorbox}
\begin{center}
	\color{blue}\underline{Lời giải:}
\end{center}
Với $X$ và $Y$ độc lập, trước hết ta tính: \begin{align*}
	\text{Var}(\overline X-\overline Y)&=\text{Var}(\overline X)+\text{Var}(\overline Y)\\
	&=\frac{1}{n^2}\text{Var}\left(\sum_{i=1}^nX_i\right)+\frac{1}{m^2}\text{Var}\left(\sum_{i=1}^mY_i\right)\\
	&=\frac{1}{n^2}\sum_{i=1}^n\text{Var}(X_i)+\frac{1}{m^2}\sum_{i=1}^m\text{Var}(Y_i)\\
	&=\frac{1}{n^2}\sum_{i=1}^n\sigma_1^2+\frac{1}{m^2}\sum_{i=1}^m\sigma_2^2\\
	&=\frac{n\sigma_1^2}{n^2}+\frac{m\sigma_2^2}{m^2}=\frac{\sigma_1^2}{n}+\frac{\sigma_2^2}{m}
\end{align*}
Vậy ta có SE$(\overline X-\overline Y)=\sqrt{\text{Var}(\overline X-\overline Y)}=\sqrt{\dfrac{\sigma_1^2}{n}+\dfrac{\sigma_2^2}{m}}$
\vspace{3mm}
\begin{tcolorbox}[enhanced,colback=blue!5!white,colframe=blue!75!black,sharp corners=all,shadow={0mm}{0mm}{-1.5mm}%
{fill=blue!75!red,opacity=0.3}]
\color{red}b) \color{black}Giả sử hai tổng thể có cùng phương sai, nghĩa là $\sigma_1^2=\sigma_2^2=\sigma^2$. Chứng minh rằng: $$S_p^2=\dfrac{(n-1)S_1^2+(m-1)S_2^2}{n-1+m-1},~~~n,m\ge2$$
là một ước lượng không chệch của $\sigma^2$.
\end{tcolorbox}
\begin{center}
	\color{blue}\underline{Lời giải:}
\end{center}
Ta có $\mathbb E(S_1^2)=\sigma_1^2=\sigma^2,~\mathbb E(S_2^2)=\sigma_2^2=\sigma^2$ và: \begin{align*}
	\mathbb E(S_p^2)&=\mathbb E\left[\frac{(n-1)S_1^2+(m-1)S_2^2}{n-1+m-1}\right]=\frac{(n-1)\mathbb E(S_1^2)+(m-1)\mathbb E(S_2^2)}{n+m-2}\\
	&=\frac{(n-1)\sigma^2+(m-1)\sigma^2}{n+m-2}=\frac{(n-1+m-1)\sigma^2}{n+m-2}\\
	&=\frac{(n+m-2)\sigma^2}{n+m-2}=\sigma^2
\end{align*}
$\Rightarrow\text{bias}(S_p^2)=\mathbb E(S_p^2)-\sigma^2=\sigma^2-\sigma^2=0$\\\\
Vậy $S_p^2$ là một ước lượng không chệch cho $\sigma^2$.
\end{document}